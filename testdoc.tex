\documentclass[12pt,a4paper,notitlepage]{article}
%\usepackage[top=20mm, bottom=20mm, left=20mm, right=40mm,showframe]{geometry}
\usepackage{testpackage}
\setromanfont{Fira Sans}
\setmainfont{Fira Sans}
\setlength{\marginparsep}{10pt}
\setlength{\marginparwidth}{20pt} 
\setlength{\textwidth}{0.9\paperwidth} 
\setlength{\hoffset}{-60pt} 
\everymath\expandafter{\the\everymath\rm}
\everydisplay\expandafter{\the\everydisplay\rm}

\usepackage{amsfonts}   % if you want the fonts
\renewcommand{\mathbb}[1]{I\!#1}


\pagestyle{fancy}
%\setlength{\headheight}{90.0pt} % Μέγεθος του header
\fancyhf{} % sets both header and footer to nothing
\renewcommand{\headrulewidth}{1pt}% Eξαφανίζει την πάνω γραμμή
\cfoot{
\begin{tikzpicture}[x=1pt,y=1pt,scale=1]
\draw (0,0) rectangle (540,11) ;
\draw (440,0)--(440,11);
	\node at(220,5)[scale=0.9,xshift=0,yshift=0,align=center]
	{\footnotesize{\textbf{TA ΘΕΜΑΤΑ ΠΡΟΟΡΙΖΟΝΤΑΙ ΓΙΑ ΑΠΟΚΛΕΙΣΤΙΚΗ ΧΡΗΣΗ ΤΗΣ ΦΡΟΝΤΙΣΤΗΡΙΑΚΗΣ ΜΟΝΑΔΑΣ}
	}};
	\node at(490,5)[scale=0.9,xshift=0,yshift=0,align=center]
	{\footnotesize{\textbf{ΣΕΛΙΔΑ \thepage ΑΠΟ \pageref{LastPage}}
		}};
\end{tikzpicture}
 	}%
\lhead{\rule{\linewidth}{1pt}\\ \hspace{1cm} }
% \lhead{Δημήτρης Τσορτανίδης}
 \chead{Επαναληπτικα Θέματα}
 \rhead{\thepage}
\pagecolor{white}
\begin{document}
		\section{ΘΕΜΑ Α}
		\begin{enumerate}[label=\textbf{\color{magenta}\thesection\greekalpha*)}]
			\item Να αποδείξετε ότι η συνάρτηση $f(x)=x^α$, $α\in\mathbb{R-Z}$ έχει παράγωγο $f'(x)=α x^{α-1}$. \monades{9}
			
			\item Να διατυπώσετε το θεώρημα Μέγιστης-Ελάχιστης τιμής.\monades{5}
			
			\item Να χαρακτηρίσετε την παρακάτω πρόταση ως αληθή ή ψευδή, αιτιολογώντας την απάντησή σας:
			
			\begin{quote}
				\textit{"Αν μία συνάρτηση είναι συνεχής, τότε είναι κατ'ανάγκη και παραγωγίσιμη."}  
			\end{quote}
			\strut \hfill \textbf{Μονάδες 3}
			
			\item Να χαρακτηρίσετε τις παρακάτω προτάσεις ως Αληθείς ή Ψευδείς:	\marginpar{\tiny \textbf{Δοκιμή $\int f(x) \:dx$}}
			\begin{enumerate}[label=\greekalpha*)]
				\item $(α^x)'=x α^{x-1}$
				
				\item	$(ln|x|)'=\dfrac{1}{x}$
				
				\item Αν οι συναρτήσεις $f,g$ έχουν όριο στο $x_0$ και ισχύει $f(x)< g(x)$ κοντά στο $x_0$, τότε $\lim\limits_{x\to x_0}f(x)<\lim\limits_{x\to x_0}g(x)$
				
				\item Αν η συνάρτηση $f$ δεν έχει ρίζες στο διάστημα Δ, τότε διατηρεί πρόσημο στο Δ.\monades{8}
			\end{enumerate}
			
		\end{enumerate}
	
	\section{\color{mybluecolour} TEST}
		\color{mybluecolour} 
	fadsfasdfasdf
	
abcdefghijklmnopqrstuvwxyzabcdefghijklmnopqrstuvwxyzabcdefghijklmnopqrstuvwxyz
\end{document}