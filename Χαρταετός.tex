\documentclass[11pt]{article}

\usepackage[
%	top=20mm,
%	bottom=20mm,
%	left=20mm,
%	right=20mm,
%	includehead
a4paper,
%scale={0.8,0.5},
%includeheadfoot,
includefoot,
ignoremp,
%ignorefoot,
layouthoffset=0pt,
top=3.5cm,
bottom=1.25cm,
footskip=1.25cm,
headsep=2cm,
scale=0.7,ignorehead,
headheight=21pt
%showframe
]{geometry}
\usepackage{array}
\usepackage{graphicx}
\usepackage{color}

%\includegraphics[height=3cm]{example-image.pdf}%

\usepackage{testpackage}
\usepackage{pgfplots}
\setromanfont{Fira Sans}
\setmainfont{Fira Sans}
\everymath\expandafter{\the\everymath\rm}
\everydisplay\expandafter{\the\everydisplay\rm}
\pagestyle{fancy}
\fancyheadoffset[]{50pt}
\lhead{\rule{\linewidth}{2pt}\\ \hspace{1cm} }
% \lhead{Δημήτρης Τσορτανίδης}
\chead{ΔΗΜΗΤΡΗΣ ΤΣΟΡΤΑΝΙΔΗΣ ΜΑΘΗΜΑΤΙΚΟΣ MSc}
\rhead{2/2018}
%\setlength{\textwidth}{0.6\paperwidth} 

\usepackage{xgreek}
\usepackage{setspace}
\usepackage{pdfpages}
\renewcommand{\baselinestretch}{1.5} 
\usepackage{multicol}% \setlength{\columnsep}{1cm}
\usepackage[export]{adjustbox}%Βάζει το πάνω μέρος της εικόνας στην αρχή της λίστας
%%%%%%%%%%%%%%%%%%%%%%%%%%
%\usepackage{showframe}
%%%%%%%%%%%%%%%%%%%%%%%%%
\usepackage{endnotes}
					\renewcommand{\makeenmark}{\textbf{\theenmark. }}
					\renewcommand{\notesname}{Απαντήσεις \hrule}
					\newcommand{\solution}[1]{\endnotetext[\value{enumi}]{#1}}
\begin{document}

\begin{center}
\huge{ΧΑΡΤΑΕΤΟΣ }

\end{center}
\rule{\textwidth}{2pt}

\section*{Εισαγωγή}
Μέρες που είναι ασδφ
\begin{center}
\begin{tikzpicture}[thick]
\node[above left] at (0,0) {O};
\node[above] at (0,2) {A};
\node[left] at (-2,0) {B};
\node[below] at (0,-3){Γ};
\node[right] at (2,0) {Δ};
\coordinate(A) at (0,2);
\coordinate(B) at (-2,0) ;
\coordinate (C) at (0,-3);
\coordinate (D) at (2,0);
\draw[||-||] (3,2)--(3,-3) node[midway,xshift=10,rotate=0,below]{$y$};
\draw[||-||] (-2,-4)--(2,-4) node[midway,below]{$x$};
\draw (A)--(C);
\draw (B)--(D);
\draw (A)--(D)node[midway,xshift=10]{$α$};\draw (A)--(B) node[midway,xshift=-10]{$α$};\draw (B)--(C)node[midway,xshift=-10]{$β$};\draw (C)--(D)node[midway,xshift=10]{$β$};
\draw (0,0.2)--(0.2,0.2)--(0.2,0);
\end{tikzpicture}
\end{center}
\section*{Διατύπωση του προβλήματος}
\begin{minipage}[b]{0.6\textwidth}Έστω ένας χαρταετός και $x$ και $y$ τα μήκη των διαγωνίων του. Αν είναι γνωστή η περίμετρος του χαρταετού, ποια είναι τα μήκη των διαγωνίων ώστε το εμβαδόν του να είναι μέγιστο;\end{minipage}
\begin{minipage}[m]{0.4\textwidth}\flushright
	\begin{tikzpicture}[baseline=(current bounding box.west),thick,scale=0.8]
	\node[above left] at (0,0) {O};
	\node[above] at (0,2) {A};
	\node[left] at (-2,0) {B};
	\node[below] at (0,-3){Γ};
	\node[right] at (2,0) {Δ};
	\coordinate(A) at (0,2);
	\coordinate(B) at (-2,0) ;
	\coordinate (C) at (0,-3);
	\coordinate (D) at (2,0);
	\draw[||-||] (3,2)--(3,-3) node[midway,xshift=10,rotate=0,below]{$y$};
	\draw[||-||] (-2,-4)--(2,-4) node[midway,below]{$x$};
	\draw (A)--(C);
	\draw (B)--(D);
	\draw (A)--(D)node[midway,xshift=10]{$α$};\draw (A)--(B) node[midway,xshift=-10]{$α$};\draw (B)--(C)node[midway,xshift=-10]{$β$};\draw (C)--(D)node[midway,xshift=10]{$β$};
	\draw (0,0.2)--(0.2,0.2)--(0.2,0);
	\end{tikzpicture}
\end{minipage} 
\section*{Λύση}
Το εμβαδον του χαρταετού δίνεται από το γνωστό τύπο 
\[Ε=\dfrac{1}{2}\cdot x\cdot y\]

H $AΓ$ είναι μεσοκάθετος της $ΒΔ$ οπότε το τρίγωνο $ΑΟΒ$ είναι ορθογώνιο. Aπό το Πυθαγόρειο θεώρημα \[ΑΟ^2+ΒΟ^2=ΑΒ^2\implies AO^2=AB^2-BO^2\implies AO=\sqrt{AB^2-BO^2}=\sqrt{α^2-\left(\dfrac{x}{2}\right)^2 }\] 
Όμοια στο τρίγωνο $ΒΟΓ$ \[OΓ=\sqrt{β^2-\left(\dfrac{x}{2}\right)^2 }\]
Από τα παραπάνω προκύπτει ότι η δεύτερη διαγώνιος είναι  \[y=AΟ+ΟΓ=\sqrt{α^2-\left(\dfrac{x}{2}\right)^2 }+\sqrt{β^2-\left(\dfrac{x}{2}\right)^2 }\]
Το εμβαδόν του χαρταετού μπορεί να εκφραστεί πλέον ως συνάρτηση του μήκους $x$ της διαγωνίου $ΒΔ$:
\[Ε(x)=\dfrac{1}{2}x\left(
\sqrt{α^2-\left(\dfrac{x}{2}\right)^2 }+\sqrt{β^2-\left(\dfrac{x}{2}\right)^2 }
\right)\]
με $x<min(2α,2β)$, λόγω τριγωνικής ανισότητας.\\
H $E(x)$ είναι παραγωγίσιμη με 
\begin{equation*} 
\begin{split}
Ε'(x) & =\dfrac{1}{2}\left(
\sqrt{α^2-\left(\dfrac{x}{2}\right)^2 }+\sqrt{β^2-\left(\dfrac{x}{2}\right)^2 }
\right)+\dfrac{1}{2}x\left(\dfrac{-x}{2\sqrt{α^2-\left(\dfrac{x}{2}\right)^2 }}+\dfrac{-x}{2\sqrt{β^2-\left(\dfrac{x}{2}\right)^2 }}\right) \\
& = 
\dfrac{1}{2}\left(
\sqrt{α^2-\left(\dfrac{x}{2}\right)^2 }+\sqrt{β^2-\left(\dfrac{x}{2}\right)^2 }
\right)
-\dfrac{1}{2}x^2
\dfrac{\sqrt{β^2-\left(\dfrac{x}{2}\right)^2 }+\sqrt{α^2-\left(\dfrac{x}{2}\right)^2 }}
{2\sqrt{α^2-\left(\dfrac{x}{2}\right)^2 }\cdot\sqrt{β^2-\left(\dfrac{x}{2}\right)^2 }} \\
& =\dfrac{1}{2}\left(
\sqrt{α^2-\left(\dfrac{x}{2}\right)^2 }+\sqrt{β^2-\left(\dfrac{x}{2}\right)^2 }
\right)\cdot
\left(
1-\dfrac{x^2}{4\sqrt{α^2-\left(\dfrac{x}{2}\right)^2}\cdot\sqrt{β^2-\left(\dfrac{x}{2}\right)^2 }}
\right)\\
&=\dfrac{1}{2}\left(
\sqrt{α^2-\left(\dfrac{x}{2}\right)^2 }+\sqrt{β^2-\left(\dfrac{x}{2}\right)^2 }
\right)\cdot
\left(
\dfrac{4\sqrt{α^2-\left(\dfrac{x}{2}\right)^2}\cdot\sqrt{β^2-\left(\dfrac{x}{2}\right)^2 }-x^2}{4\sqrt{α^2-\left(\dfrac{x}{2}\right)^2}\cdot\sqrt{β^2-\left(\dfrac{x}{2}\right)^2 }}
\right)\\
&=\dfrac{1}{2}\dfrac{
\sqrt{α^2-\left(\dfrac{x}{2}\right)^2 }+\sqrt{β^2-\left(\dfrac{x}{2}\right)^2 }
}{\sqrt{α^2-\left(\dfrac{x}{2}\right)^2}\cdot\sqrt{β^2-\left(\dfrac{x}{2}\right)^2 }}\cdot
\left(
\sqrt{α^2-\left(\dfrac{x}{2}\right)^2}\cdot\sqrt{β^2-\left(\dfrac{x}{2}\right)^2 }-\left(\dfrac{x}{2}\right)^2
\right)
\end{split}
\end{equation*}
Από την παραπάνω $$\dfrac{1}{2}\dfrac{
	\sqrt{α^2-\left(\dfrac{x}{2}\right)^2 }+\sqrt{β^2-\left(\dfrac{x}{2}\right)^2 }
}{\sqrt{α^2-\left(\dfrac{x}{2}\right)^2}\cdot\sqrt{β^2-\left(\dfrac{x}{2}\right)^2 }}>0$$
και για $0<x<min(2α,2β)$
\begin{equation*} 
\begin{split}
Ε'(x)=0\iff\sqrt{α^2-\left(\dfrac{x}{2}\right)^2}\cdot\sqrt{β^2-\left(\dfrac{x}{2}\right)^2 }-\left(\dfrac{x}{2}\right)^2&=0\iff\\
\sqrt{α^2-\left(\dfrac{x}{2}\right)^2}\cdot\sqrt{β^2-\left(\dfrac{x}{2}\right)^2 }&=\left(\dfrac{x}{2}\right)^2\iff\\
\left(α^2-\left(\dfrac{x}{2}\right)^2\right)\cdot\left(β^2-\left(\dfrac{x}{2}\right)^2\right)&=\left(\dfrac{x}{2}\right)^4\iff\\
\left(\dfrac{x}{2}\right)^4-\left(α^2+β^2\right)\left(\dfrac{x}{2}\right)^2+α^2β^2&=\left(\dfrac{x}{2}\right)^4\iff\\
\left(α^2+β^2\right)\left(\dfrac{x}{2}\right)^2&=α^2β^2\iff\\
\left(\dfrac{x}{2}\right)^2&=\dfrac{α^2β^2}{α^2+β^2}\overset{x>0}{\iff}\\
x&=2\sqrt{\dfrac{α^2β^2}{α^2+β^2}}=\dfrac{2αβ}{\sqrt{α^2+β^2}}
\end{split}
\end{equation*} 
Η λύση είναι δεκτή\footnote{Αν $α<β$ ισχύει: ${β^2<α^2+β^2\iff β<\sqrt{α^2+β^2}\iff \frac{β}{\sqrt{α^2+β^2}}<1\iff\dfrac{2αβ}{\sqrt{α^2+β^2}}<2α}$,\\ ομοίως αν $β<α$}.
\\Όμοια $Ε'(x)>0\iff 0<x<\dfrac{2αβ}{\sqrt{α^2+β^2}}$.
\begin{center}
	\begin{tikzpicture}
	\draw (0,4) rectangle (5,6);
	\draw (0,3) rectangle (5,4);
	\draw (2,3)--(2,6);
	\draw(0,4)--(5,4);
	\draw(0,5)--(5,5);
	\draw(3.5,3)--(3.5,5) node[above]{\scriptsize$\frac{2αβ}{\sqrt{α^2+β^2}}$};
	\node[above right,xshift=-4] at (2,5){\small$0$};
	\node[above left,xshift=31] at (5,5){\scriptsize$min(2α,2β)$};
	\node at (1,4.5){$E'(x)$};\node at (2.75,4.5){+};\node at (4.25,4.5){--};\node at (3.5,4.5){0};\node at (3.5,2.7){\scriptsize max};
	\node at (1,3.5){$E(x)$};\node at (2.75,3.5){$\nearrow$};\node at (4.25,3.5){$\searrow$};
	\node [above]at (0.5,5){$x$};
	\end{tikzpicture}
\end{center}
\[\begin{split}
y&=\sqrt{α^2-\left(\dfrac{x}{2}\right)^2 }+\sqrt{β^2-\left(\dfrac{x}{2}\right)^2 }=
\sqrt{α^2-\left(\dfrac{2αβ}{2\sqrt{α^2+β^2}}\right)^2}+\sqrt{β^2-\left(\dfrac{2αβ}{2\sqrt{α^2+β^2}}\right)^2}\\&=
\sqrt{α^2-\dfrac{α^2β^2}{α^2+β^2}}+\sqrt{β^2-\dfrac{α^2β^2}{α^2+β^2}}=
\sqrt{\dfrac{α^4+α^2β^2-α^2β^2}{α^2+β^2}}+\sqrt{\dfrac{β^4+α^2β^2-α^2β^2}{α^2+β^2}}\\&=
\dfrac{α^2}{\sqrt{α^2+β^2}}+\dfrac{β^2}{\sqrt{α^2+β^2}}=\dfrac{α^2+β^2}{\sqrt{α^2+β^2}}=\sqrt{α^2+β^2}
\end{split}
\]
Οπότε οι διαγώνιοι πρέπει να έχουν μήκη:\textbf{\begin{itemize}
	\item $\mathbf{x=\dfrac{2αβ}{\sqrt{α^2+β^2}}}$
	\item $\mathbf{y=\sqrt{α^2+β^2}}$
\end{itemize}}
\end{document}